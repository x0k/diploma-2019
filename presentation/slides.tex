%%% Содержимое слайдов

% Титульный слайд
\frame[plain]{\titlepage}

%-------------------------------------------------------------------------------

\section{Введение}

\begin{frame}
\frametitle{\insertsection}
\framesubtitle{Цели и задачи}

\textbf{Цель:}
\begin{itemize}
    \item Проектирование приложения для представления комплексных расписаний
\end{itemize}

\textbf{Задачи:}
\begin{itemize}
    \item Анализ предметной области
    \item Определение способа реализации процесса
    \item Проектирование архитектуры приложения
    \item Реализация приложения
\end{itemize}

\end{frame}

%-------------------------------------------------------------------------------

\section{Основные понятия}

\begin{frame}
\frametitle{\insertsection}
\framesubtitle{Веб-платформа}

\textbf{Преимущества:}
\begin{itemize}
    \item Распространенность
    \item Кроссплатформенность
    \item Опыт разработки
\end{itemize}

\textbf{Недостатки:}
\begin{itemize}
    \item Производительность
    \item Ограниченность среды исполнения кода
\end{itemize}

\end{frame}

%-------------------------------------------------------------------------------

\section{Основные понятия}

\begin{frame}
\frametitle{\insertsection}
\framesubtitle{Комплексное расписание}

\begin{figure}
    \center
    \includegraphics[height=60mm]{schedule_types}
\end{figure}
\end{frame}

%-------------------------------------------------------------------------------

\section{Основные понятия}

\begin{frame}
\frametitle{\insertsection}
\framesubtitle{Пример комплексного расписания}

\begin{figure}
    \center
    \includegraphics[width=\linewidth]{monday}
\end{figure}
\end{frame}

%-------------------------------------------------------------------------------

\section{Основные понятия}

\begin{frame}
\frametitle{\insertsection}
\framesubtitle{Представление расписания}

\begin{figure}
    \center
    \includegraphics[height=60mm]{tasks}
\end{figure}
\end{frame}

%-------------------------------------------------------------------------------

\section{Что такое расписание?}

\begin{frame}
\frametitle{\insertsection}

\begin{figure}
    \center
    \includegraphics[width=80mm]{date_mapper}
\end{figure}
\end{frame}

%-------------------------------------------------------------------------------

\section{Представление расписания}

\begin{frame}[fragile]
\frametitle{\insertsection}
\framesubtitle{Генерация функции}

\minipage{0.35\textwidth}
    \begin{lstlisting}[basicstyle=\tiny]
{
    "data": {
        "subject": "Math",
        "type": "Lab",
        "teacher": "Ivan",
        "room": "123"
    },
    "options": {
        "includes": {
            "day": [1, 3]
        }
    },
    "rules": {
        "includes": {
            "call": [6, 7]
        }
    }
}
    \end{lstlisting}
\endminipage\hfill
\minipage{0.6\textwidth}
    \begin{lstlisting}[basicstyle=\tiny]
[
    "@case", [
        "@and", [
            "$>>", "@get", [ "day" ], "@includes", [ [1, 3] ],
            "$>>", "@get", [ "call" ], "@includes", [ [6, 7] ]
        ], "123",
        "@default", [ false ] ],
]
    \end{lstlisting}
\endminipage

\end{frame}

%-------------------------------------------------------------------------------

\section{Представление расписания}

\begin{frame}[fragile]
\frametitle{\insertsection}
\framesubtitle{Временной интервал}

\begin{figure}
    \center
    \includegraphics[height=60mm]{time_structure}
\end{figure}
\end{frame}

%-------------------------------------------------------------------------------

\section{Представление расписания}

\begin{frame}
\frametitle{\insertsection}
\framesubtitle{Генератор событий}

\vspace{1cm}

\begin{figure}
    \center
    \includegraphics[width=\linewidth]{event_build_order}
\end{figure}
\end{frame}

%-------------------------------------------------------------------------------

\section{Представление расписания}

\begin{frame}
\frametitle{\insertsection}
\framesubtitle{Группировщик событий}

\vspace{1cm}

\begin{figure}
    \center
    \includegraphics[width=\linewidth]{grouper}
\end{figure}
\end{frame}

%-------------------------------------------------------------------------------

\section{Применение}

\begin{frame}
\frametitle{\insertsection}

\begin{figure}[!htb]
    \minipage{0.61\textwidth}
        \includegraphics[width=\linewidth]{client_schedule}
    \endminipage\hfill
    \minipage{0.34\textwidth}
        \includegraphics[width=\linewidth]{client_mobile}
    \endminipage
\end{figure}

\end{frame}

%-------------------------------------------------------------------------------

\section{Спасибо за внимание}

\begin{frame}
    \frametitle{\insertsection}

    \textbf{Тема:}
    \begin{itemize}
        \item Проектирование приложения для представления комплексных расписаний с использованием подхода Progressive Web Application
    \end{itemize}

    \textbf{Автор:}
    \begin{itemize}
        \item Красильников Роман Борисович
    \end{itemize}

\end{frame}
