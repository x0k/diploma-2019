\section{Анализ проблемы}

\subsection{Анализ предметной области}

Для исследования предметной области было выбрано расписание занятий студенческой группы 147 ФГБОУ ВО <<СГУ им. Питирима Сорокина>> на первый семестр 2018/2019 учебного года  представленное на рисунке \ref{schedule-147}.

\addimgh{schedule_147}{0.66}{Расписание 147 группы}{schedule-147}

Расписание представляет собой таблицу, состоящую из следующих столбцов (таблица \ref{tbl:schedule-header}).

\begin{tbl}{Шапка расписания}{tbl:schedule-header}
  \begin{tabularx}{\textwidth}{| *{3}{c |} X |}
  \hline День недели & Номер пары & Время & Предмет (наименование, тип, преподаватель, кабинет, ограничения) \\
  \hline
  \end{tabularx}
\end{tbl}

Рассмотрев приведенное расписание, можно сделать вывод, что последний столбец перегружен информацией относительно остальных столбцов.
Помимо этого он может разделятся на дополнительные столбцы в зависимости от количества подгрупп в группе.

В свою очередь конкретные ячейки в столбце предмета могут дробиться горизонтально на ячейки с <<режимами>> - числитель и знаменатель.
В момент начала, действия расписания начинается отсчет недель, где недели с нечетным номером являются числителем, а с четным - знаменателем.

Основные типы ограничений, которые могут дополнительно возникать в расписании:
\begin{easylist}
  & Предмет начинается с конкретной даты;
  & Предмет оканчивается конкретной датой;
  & Предмет преподается в конкретный период;
  & Предмет не преподается по конкретным датам;
  & Предмет преподается только по конкретным датам.
\end{easylist}

Представленный выше набор атрибутов позволяет организовывать комплексные расписания периодично (на протяжении времени действия расписания, оно повторяется с некоторым периодом).
Таким образом конечным видом расписания становится - таблица.

С другой стороны, периодичность влечет информационную перегруженность большинства полей и необходимость дополнительных вычислений (числитель, знаменатель).

Один из способов увеличить репрезентативность комплексного расписания - изменить подход к его представлению.
Например в виде упорядоченного по времени списка, макет которого представлен на рисунке \ref{img:tasks}.

\addimgh{tasks}{0.4}{Представление расписания в виде списка}{img:tasks}

Данное представление обладает всей необходимой информацией для студента подгруппы <<а>> на 26.11.2018 число.
А также избавлено от информационной перегруженности и нерелевантной информации.

В рамках проекта сформируем необходимый набор терминов и понятий:

Расписание - упорядоченная по времени информация о предстоящих событиях.

Комплексное расписание - расписание, которое не может быть выражено в периодической форме без ввода некоторых условностей (режимы недель <<числитель>> и <<знаменатель>>).

Событие - ограниченное во времени неизменное положение вещей.
Например событие <<пара>> может быть описано следующим способом:
\begin{easylist}
  & Предмет: проектирование информационных систем;
  & Тип пары: лекция;
  & Преподаватель: Иванов И. И;
  & Место проведения: 427;
  & Время: 18.12.2018 11:20 - 18.12.2018 12:50.
\end{easylist}

Состояние (расписания) - событие наблюдаемое на конкретный момент времени.

\subsection{Анализ текущего состояния}

На текущий момент наиболее распространенные сценарии использования расписания среди студентов ФГБОУ ВО <<СГУ им. Питирима Сорокина>> в порядке убывания актуальности могут выглядеть следующим образом:

\begin{easylist}
  & Просмотр необновляемой цифровой копии на мобильном устройстве (фотография, *.pdf или иной файловый формат, расположенный на локальном носителе);
  & Просмотр расписания с мобильного устройства на web-сайте или каком-либо сервере, которое может обновляться уполномоченным на это сотрудником учебной организации (будем называть его далее  <<актуальным>>);
  & Просмотр необновляемой цифровой копии с ПК;
  & Просмотр актуального расписания на доске расписаний;
  & Просмотр актуального расписания с ПК.
\end{easylist}

Экспресс-опрос студентов показал, что наиболее актуальным сценарием использованием является применение мобильных платформ.

\subsubsection{Модель процесса}

Рассмотрим AS-IS модель процесса планирования с использованием комплексного расписания представленную на рисунке \ref{img:as-is}.

\addimg{as_is}{0.8}{Модель процесса <<Планирование с использованием комплексного расписания>>, AS-IS}{img:as-is}

Из данной модели видно, что в процессе планирования возникают дополнительные вычисления.
Для приведенного в качестве примера расписания - это определение режима недели и наложение ограничений предмета на выбранный период.

Подобные действия в действительности являются раздражающими и часто становятся причинами различных ошибок.

\subsection{Существующие решения}

На сегодняшний день существует приложение, позволяющее решать описанную проблему.

\subsubsection{SHEView}

SHEView - приложение для операционной системы Android, которое позволяет представлять комплексное расписание в виде списка <<событий>>, сгруппированных по датам.
Интерфейс приложения представлен на рисунке \ref{img:sheview}.

\begin{figure}[!tb]
  \minipage{0.32\textwidth}
      \includegraphics[width=\linewidth]{sheview}
  \endminipage\hfill
  \minipage{0.32\textwidth}
      \includegraphics[width=\linewidth]{sheview_menu}
  \endminipage\hfill
  \minipage{0.32\textwidth}
      \includegraphics[width=\linewidth]{sheview_groups}
  \endminipage
  \caption{Интерфейс программы SHEView} \label{img:sheview}
\end{figure}

Данное приложение позволяет просматривать комплексное расписание в виде списка событий, сгруппированных по дням.
Кроме того, имеется доступ к справочникам: преподаватели и предметы.
Поскольку приложение ориентировано на работу с разделяемыми расписаниями, в нем реализована возможность выбора подгруппы.

\subsubsection{SHEMaker}

Создание файла расписания происходит в специальной программе - <<SHEMaker>>, интерфейс которой представлен на рисунке \ref{img:shemaker}.

\addimg{shemaker}{0.8}{Окно ввода правил расписания SHEMaker}{img:shemaker}

Создание расписания начинается с заполнения окна настроек (рисунок \ref{img:shemaker-settings}), в зависимости от выбранных флагов на данном этапе, далее будут доступны от 2 до 5 экранов ввода справочной информации: график занятий, режимы, группы, преподаватели, занятия.

\addimg{shemaker_settings}{0.8}{Окно настроек расписания SHEMaker}{img:shemaker-settings}

После переключения флага в правом верхнем углу (рисунок \ref{img:shemaker}), программа переходит в режим ввода правил расписания, становятся доступны семь окон ввода, по одному на день недели, в каждом окне находится таблица со столбцами, представленными в таблице \ref{tbl:schedule-editor-header}.

\begin{tblh}{Шапка таблицы ввода правил расписания}{tbl:schedule-editor-header}
  \begin{tabularx}{\textwidth}{| *{10}{X |}}
    \hline Номер
    & Исклю--чение
    & Время
    & Группа
    & Режим
    & Преп--одава--тель
    & Предм--ет
    & Место
    & Начало
    & Конец \\
    \hline
  \end{tabularx}
\end{tblh}

Строка изображенной таблицы преимущественно состоит из полей - выпадающих списков, содержащих справочную информацию, вносимую на предыдущем этапе.
Поле место заполняется как текстовое, исключением маркируются строки, которые на этапе вычисления будут отсеивать события, с эквивалентными параметрами.

Для хранения и передачи информации о расписании между приложениями используется разработанная структура \hyperlink{json}{JSON} файла, отношения в которой можно выразить в виде ER-диаграммы изображенной на рисунке \ref{img:er-diagram}.

\addimgh{erd}{0.8}{Структура файла расписания}{img:er-diagram}

Дадим описание некоторых используемых сущностей (остальные достаточно очевидны):

\begin{easylist}
  & schedule - расписание
  && date\_from, date\_to - даты начала и окончания расписания
  && modes\_enabled - использовать данные из поля mode
  && groups\_enabled - использовать данные из поля group
  && teachers\_enabled - использовать данные из поля teacher
  & mode - режим, является подмножеством цикла режимов (для расписания, приведенного в качестве примера, цикл составляет 2 недели).
  && from - начало периода
  && length - продолжительность периода
  && name - наименование периода
  & event - событие
  && day - день недели
  && exception - исключение, в ходе процесса построения расписание событие с указанными атрибутами будет исключено из потока событий
  && place - строковое поле, для указания места проведения
\end{easylist}

Представленная модель данных позволяет описывать различные вариации учебных расписаний, как для студенческих групп, так и для преподавателей.
Для представления простых расписаний имеется возможность не использовать функционал режимов.

\subsubsection{Выводы по существующему решению}

Приложения <<SHEView>> и <<SHEMaker>> могут быть использованы в процессе представления комплексных расписаний, однако они имеют ряд объективных недостатков:

\begin{easylist}
  & Малый охват платформ;
  & Необходимость обработки актуального расписания человеком во время создания файла расписания;
  & Конечный набор атрибутов события и простая модель режимов.
\end{easylist}

Исходя из вышеизложенных недостатков, можно заключить, что данное решение не может называться универсальным.

\subsection{Предлагаемое решение}

Существует несколько способов решения проблемы с необходимостью дополнительных вычислений:
\begin{easylist}[enumerate]
  & Не использовать вычисляемые данные в расписании;
  & Автоматизация дополнительных вычислений и выборки событий.
\end{easylist}

Первый способ приведет расписание к классу <<простых>> и его неминуемому <<раздуванию>>.
Данное решение, возможно, даже несколько ухудшит ситуацию.
Поэтому будем рассматривать второй вариант решения проблемы.

Для применения автоматизации необходимо понять в чем заключается принципиальная разница между вычисляемыми данными и выборкой событий.

Для человека существует набор <<очевидных>> данных, таких как время и дата, для них не используются вычисления как таковые.
Однако такие параметры, как текущий режим недели с точки зрения человека, нельзя отнести к <<очевидным>>.
В свою очередь, для компьютера операция по определению текущей даты и времени не отличаются от определения текущего режима недели.

Из этого следует, что в автоматизированном процессе выделение вычисления метаданных как отдельного этапа - не рационально.
Используя подход автоматизации построим модель процесса TO-BE (рисунок \ref{img:to-be}).

\addimgh{to_be}{0.8}{Модель процесса <<планирование с использованием комплексного расписания>>, TO-BE}{img:to-be}

Из представленной модели, следует, что операции по вычислению данных и выбору событий из расписания берет на себя приложение, представляя план событий для осуществления планирования.

\subsection{Анализ требований}

Определим стейкхолдеров для текущего проекта (таблица \ref{tbl:stackeholders}).

\begin{tbl}{Матрица стейкхолдеров}{tbl:stackeholders}
  \begin{tabularx}{\textwidth}{| p{1.6cm} | p{1.5cm} | X | p{1.5cm} | X | X |}
  \hline Название
  & Статус
  & Ожидания от проекта
  & Степень влияния
  & Связанные риски
  & Стратегия \\
  \hline Красиль--ников Роман
  & Исполни--тель
  & Успешное достижение поставленных целей
  & 10
  & Недостаточная компетентность \newline Безответстве--нность
  & Постоянное поддержание контакта \newline Установка дедлайнов \\
  \hline Студенты и преподаватели
  & Пользо--ватель
  & Работоспосо--бный продукт
  & 4
  & Незаинтересова--нность \newline Сопротивление
  & Тесное взаимодействие \newline Демонстрация продукта \\
  \hline Разрабо--тчики
  & Пользо--ватель
  & Необходимый функционал
  & 3
  & Высокие требования \newline Незаинтересова--нность
  & Разработка документации \\
  \hline
  \end{tabularx}
\end{tbl}

Сформулируем общие требования от лица пользователя к процессу автоматизации <<планирования на основе комплексного расписания>>:

\begin{easylist}[itemize]
  & Добавление комплексного расписания в систему;
  && Путем извлечения данных из существующего расписания;
  && Путем комбинации расписаний;
  & Просмотр комплексного расписания:
  && Выбор способа представления;
  && Выбор периода представления;
  & Удаление расписания;
  & Обновление расписания.
\end{easylist}

Требования от лица разработчиков:

\begin{easylist}
  & Дружественный программный интерфейс;
  & Эффективная работа алгоритмов;
  & Минимальное количество рутинных операций.
\end{easylist}

Функциональные требования можно представить в виде UseCase диаграммы изображенной на рисунке \ref{img:use-case}.

\addimg{use_case}{1}{UseCase диаграмма}{img:use-case}

Также выделим некоторые нефункциональные требования к системе.
\begin{easylist}
  & Реализация для всех платформ - приложение может быть запущено на всех актуальных платформах;
  & Возможность работы без доступа к сети.
\end{easylist}

\clearpage