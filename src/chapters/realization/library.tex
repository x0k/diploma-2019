\subsection{Реализация библиотеки}

Для обеспечения процесса представления расписания описанного в главе <<Проектирование>> необходимо разработать следующий функционал:
\begin{easylist}[enumerate]
  & Конвертации транзитного формата в базу правил;
  & Создания временного периода с учетом временных ограничений;
  & Вычисления состояний на временном промежутке;
  & Преобразование состояний в список событий.
\end{easylist}

\subsubsection{Конвертер расписаний}

Изначально для создания шаблонов правил необходим набор нативных операторов, комбинацией которых и будут получатся шаблоны.
Данные операторы должны быть чистыми функциями \cite{clear-functions} и достаточно примитивными, что бы описывать разнообразные правила.
Конечный список операторов представлен в Приложении \hyperlink{app:C}{В}.

Стоит подробно остановится на способе применения операторов <<case>> и <<default>>.
Данные операторы позволяют выбирать из набора условий или возвращать значение по умолчанию, если ни одно из условий не было выполнено.

Пример использования операторов приведен в листинге \ref{lst:case-operator}.

\begin{lstlisting}[caption={Пример использования операторов <<case>> и <<default>>},label={lst:case-operator}]
'$>>', '@dateTime', ['time'],
"@case", ["@in", [{ "start": -2208969017000, "end": -2208963617000 }], 1,
  "@case", ["@in", [{ "start": -2208963017000, "end": -2208957617000 }], 2,
    "@case", ["@in", [{ "start": -2208957017000, "end": -2208951617000 }], 3,
      "@case", ["@in", [{ "start": -2208948017000, "end": -2208942617000 }], 4,
        "@case", ["@in", [{ "start": -2208942017000, "end": -2208936617000 }], 5,
          "@case", ["@in", [{ "start": -2208936017000, "end": -2208930617000 }], 6,
            "@case", ["@in", [{ "start": -2208930017000, "end": -2208924617000 }], 7,
              "@case", ["@in", [{ "start": -2208924017000, "end": -2208918617000 }], 8,
                "@default", [false] ] ] ] ] ] ] ] ]

const action = buildAction(expression)
const date = { minute: 0, hour: 18, day: 6, month: 4, year: 2019 }

action(date) // 6
\end{lstlisting}

С помощью данного выражения вычисляется номер текущей пары.
Это происходит следующим образом:
\begin{easylist}
  \ListProperties(Start1=1)
  & Из комплексного аргумента функции извлекается часть даты состоящая из часов и минут;
  & Происходит ряд проверок вхождения даты в указанные временные интервалы;
  & По достижению условия возвращается полученный результат, иначе выполняется оператор <<default>>, который устанавливает значение номера пары - <<false>>.
\end{easylist}

После определения набора операторов можно переходить к реализации самого конвертера.
Таковой будет работать по следующему принципу: для каждого атрибута, перечисленного в поле <<fields>> создается правило определения его значения по наборам используемых ограничений.

Для генерации правила вычисления значения атрибута - объединяются все выражения построенные для перечисленных в поле <<events>> событий в которых используется этот атрибут.
На данный момент объединение происходит примитивным образом - выражения группируются по значениям.
В случае, когда несколько выражений вычисляют одно значение - такие выражения объединяются оператором <<or>>.

Например, для вычисления значения атрибута room указанного в событии листинга \ref{lst:event-example} будет сгенерировано правило отображенное в листинге \ref{lst:rule-example}.

\begin{lstlisting}[caption={Пример события с набором ограничений},label={lst:event-example}]
{
  "data": {
      "subject": "Предмет",
      "type": "лаб.",
      "teacher": "Преподаватель",
      "room": "251/1"
  },
  "options": {
      "includes": { "day": [1, 3] }
  },
  "rules": {
      "includes": { "call": [6, 7] }
  }
}
\end{lstlisting}

\begin{lstlisting}[caption={Пример правила вычисляющего значение атрибута <<room>>},label={lst:rule-example}]
[
  "@case", [
    "@and", [
      "$>>", "@get", [ "day" ], "@includes", [ [1, 3] ],
      "$>>", "@get", [ "call" ], "@includes", [ [6, 7] ]
    ],
    "251/1",
    "@default", [ false ] ],
]
\end{lstlisting}

Текущая реализация генерации правил - не оптимальна.
Так как встречаются ситуации, когда выражения использующиеся для вычисления одного значения атрибута могут иметь дублирующиеся условия.
Для решения таких проблем необходимо приводить логические выражения к СДНФ \cite{discrete-math}.
Однако на данном этапе разработки текущее решение является допустимым и может быть доработано позже.

\subsubsection{Генератор временных периодов}

Под генерацией временного периода понимается создание итератора предоставляющего оптимальный набор дат с точки зрения вычисления состояний и их последующего преобразования в события.
Для достижения данной цели необходимо переосмыслить итерирование временных периодов.

Рассмотрим временной период как иерархическую структуру изображенную на рисунке \ref{img:time-structure}.

\addimgh{time_structure}{0.9}{Структура временного периода}{img:time-structure}

Для осуществления гибкого контроля над тем, какие даты будут использованы для дальнейших вычислений можно на каждом из иерархических уровней установить шаг итерации.

Так как количество комбинаций увеличивается с каждым уровнем - имеет смысл производить валидацию даты как можно раньше.
Например на уровне день уже можно определить является ли данный день выходным и пропустить его, таким образом можно избежать ненужных проверок на более низких уровнях.

Для реализации данного поведения используются генераторы, реализующие независимую логику каждого уровня.
Затем используя метод <<wrapIterable>> верхний уровень оборачивается нижележащим уровнем, постепенно собирая полноценную дату (листинг \ref{lst:time-period-example}).

\begin{lstlisting}[caption={Процесс составления временного периода},label={lst:time-period-example}]
const endTime = end.getTime()
const condition = ({ milliseconds }: IMinutes) => milliseconds <= endTime
const years = withConstraints(yearsIterator, yearConstraint)(year)
const months = wrapIterable(years, withConstraints(monthsIterator, monthConstraint), month)
const dates = wrapIterable(months, withConstraints(dateIterator, dateConstraint), date)
const hours = wrapIterable(dates, withConstraints(hoursIterator, hourConstraint), hour)
const minutes = wrapIterable(hours, withConstraints(minutesIterator, minuteConstraint), minute)
return restrictIterable<IMinutes>(minutes, condition)
\end{lstlisting}

Структура ограничений представлена в таблице \ref{tbl:constraints-structure}.

\begin{tblh}{Структура временных ограничений}{tbl:constraints-structure}
  \begin{tabularx}{\textwidth}{| p{3cm} | p{4cm} | X |}
    \hline Наименование & Тип                  & Описание                \\
    \hline step?        & number | TExpression & Шаг изменения параметра \\
    \hline expression?  & TExpression          & Валидатор параметра     \\
    \hline
    \end{tabularx}
\end{tblh}

\subsubsection{Реализация вычислителя}

На данном этапе необходимо реализовать способ вычисления состояния расписания на некоторый момент времени используя правила из базы правил.

Поскольку операции используемые в правила - чистые функции, то их комбинации также чистые функции.
Тогда вычисление состояния можно представить следующим образом:
\begin{easylist}[enumerate]
  & На основе текущей даты вычисляется значение некоторого правила;
  & Значение добавляется к текущей дате таким образом, что бы значение могло бы использоваться в последующих вычислениях;
  & Повторять шаг 1 с данными полученными на шаге 2 пока есть невычисленные правила.
\end{easylist}

Для реализации вышеприведенного алгоритма необходимо отсортировать правила таким образом, что бы для каждого правила, правила от которых зависит его вычисление были бы уже вычислены.

Получив отсортированный список правил, они преобразуются в комбинации операций, и последовательно вычисляются, по итогу образуя состояние расписания на конкретный момент.

\subsubsection{Реализация преобразователя}

Процесс преобразования потока состояний в список событий может быть представлен как объединение состояний описывающих одно событие в различные моменты времени.
В процессе объединения необходимо соблюсти ряд условий:
\begin{easylist}
  \ListProperties(Start1=1)
  & Значения атрибутов состояний тождественны;
  & Временной промежуток между состояниями не превышает 1 шага из сгенерированного временного периода.
\end{easylist}

Если с первым условием все понятно, то для проверки второго необходимо вычислить минимальный шаг, исходя из временных ограничений использованных при генерации временного периода.
После процесса преобразования события будут иметь структуру представленную в таблице \ref{tbl:grouped-event-structure}.

\begin{tblh}{Структура преобразованого события}{tbl:grouped-event-structure}
  \begin{tabularx}{\textwidth}{| p{3cm} | p{4cm} | X |}
    \hline Наименование & Тип                  & Описание                 \\
    \hline value        & T extends IDateTime  & Атрибуты события         \\
    \hline period       & IPeriod<number>      & Временной период события \\
    \hline
    \end{tabularx}
\end{tblh}

\subsubsection{Применение}

Пример использования разработанной библиотеки <<eventa>> \cite{eventa} представлен в Приложении \hyperlink{app:D}{Г}.

В нем использованы все разработанные компоненты, а также указан способ группировки состояний.
Он заключается в проверке совпадения данных состояний и разницы моментов вычисления состояний, которая не должна превышать шаг вычисления состояний.

Процесс взаимодействия компонентов описан в терминах диаграммы активностей (рисунок \ref{img:activity-diagram}).

\addimgh{activity_diagram}{0.7}{Процесс представления расписания}{img:activity-diagram}
