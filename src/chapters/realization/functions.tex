\subsection{Реализация переносимых функций}

Для обеспечения переносимости функций между средами исполнения была поставлена задача разработать библиотеку, позволяющую генерировать шаблоны функций и создавать по ним нативно вычисляемые функции \cite{rule-interpreter}.

Решение данной задачи заключается в следующем: пользователь определяет набор нативных функций и составляет из них выражение в \hyperlink{json}{JSON} формате по следующим правилам.
\begin{easylist}
  & Имена функций предварены переопределяемым префиксом. По умолчанию - <<@>>
  & После имени функции следует список параметров, передаваемых в функцию по умолчанию
  & Определен набор действий влияющих на процесс интерпретации предваренных переопределяемым префиксом. По умолчанию - <<\$>>
  && <<>{}>{}>> композиция функций
  && <<eval>> вычисляет значение функции на месте
  && <<map>> в качестве аргументов принимает массив и функцию отображения, вычисляет новый массив путем поэлементного применения функции
\end{easylist}

Пример набора нативных функций представлен в листинге \ref{lst:native-functions-example}.

\clearpage

\begin{lstlisting}[caption={Пример набора нативных функций},label={lst:native-functions-example}]
const operations = {
  '-': (a: number, b: number) => a - b,
  '/': (a: number, b: number) => a / b,
  inc (step: number, value: number) {
    return value + step
  },
  sum (...numbers: number[]) {
    return numbers.reduce((sum, value) => sum + value, 0)
  }
}
\end{lstlisting}

После определения шаблонов функций и создания по ним нативных, функции можно будет вычислить.
Данный процесс и его результат представлен в листинге \ref{lst:build-actions-example}.

\begin{lstlisting}[caption={Пример создания и вычисления шаблонных функций},label={lst:build-actions-example}]
const actionsBuilder = buildActionsBuilder(operations)
function buildAction (expression: TExpression) {
  return actionsBuilder(expression)[0]
}
const expression1 = [ '$map', [1, 2, 3], '@inc', [3] ]
const expression2 = [ '$eval', '@sum', [1, 2, 3] ]
const expression3 = [ '@-', [ 10, 3 ] ]
const expression4 = [ '@/', [ 20 ] ]
const action1 = buildAction(expression1) // [4, 5, 6]
const action2 = buildAction(expression2) // 6
const action3 = buildAction(expression3)
const action4 = buildAction(expression4)
action3() // 7
action4(5) // 4
\end{lstlisting}