\subsection{Реализация клиента}

Основные моменты реализации веб-клиента.

\subsubsection{Представление событий}

По итогу реализации библиотеки <<eventa>> логика вычисления данных расписания уже готова.
Однако, остается вопрос, как представлять вычисленные данные, ведь для каждого расписания наборы данных и оптимальные способы их представления различаются.

Для решения данной проблемы используется ранее разработанная библиотека <<rule-interpreter>>, позволяющая каждому расписанию указывать способ представления событий.
Как и раньше реализован набор нативных функций позволяющих комбинировать и извлекать данные для создания React-элемента, отображаемого на странице расписания.
Список нативных функций:
\begin{easylist}
  & <<extract>> создает строку по указанному шаблону, содержащую данные извлекаемые из комплексного аргумента;
  & <<element>> создает React-элемент с указанными свойствами;
  & <<property>> создает свойство с указанным именем и значением;
  & <<compose>> n-арный комбинатор, последний аргумент которого - значение для вычисления списка функций;
  & <<flat>> - позволяет вызвать функцию с списком аргументов, указанных в виде массива.
\end{easylist}

Используя указанные функции был составлен компонент для представления событий расписания приведенного в качестве примера.
Код функции отображен в листинге \ref{lst:event-function}

\begin{lstlisting}[caption={Функция представления события},label={lst:event-function}]
"@compose", [
  "$>>", "@compose", [
    "$>>", "@extract", ["%subject"],
    "@property", ["primary"],
    "$>>", "@extract", ["%type, %teacher"],
    "@property", ["secondary"]
  ],
  "@flat", [ "@element", ["ListItemText", { "key": "item-primary" }] ],
  "$>>", "@extract", ["%room"],
  "$>>", "@property", ["children"],
  "$>>", "@element", ["Typography"],
  "$>>", "@property", ["children"],
  "@element", ["ListItemSecondaryAction", { "key": "item-secondary" }]
]
\end{lstlisting}

Список элементов доступных для создания может быть легко расширен элементами из библиотеки <<Material-UI>>, а также собственными компонентами.

\subsubsection{Произведение вычислений}

Поскольку вычисление списка событий довольно ресурсоемкий процесс, то его выполнение в основном потоке может вызвать отказ работы интерфейса на момент вычислений.
Решение данной проблемы состоит в использовании дополнительных потоков посредством создания веб-воркеров для произведения вычислений.

Использование веб-воркеров состоит в следующем:
\begin{easylist}[enumerate]
  & Применяется загрузчик веб-воркеров для файлов с определенной сигнатурой;
  & В файле воркера реализуется логика также, как и в основном коде;
  & Добавляется обработчик события <<onmessage>> принимающий начальные аргументы и начинающий вычисления. По завершению вычислений вызывается метод <<postMessage>> для отправки вычисленных данных;
  & В коде основного потока создается файл подключения воркера. В данной реализации вызов воркера оборачивается в специальный объект типа <<Promise>> для осуществления асинхронных операций.
\end{easylist}

Используя данный подход в отдельный поток были вынесены операции по вычислению событий расписания и по группировке событий.

\subsubsection{Интерфейс приложения}

Главный экран приложения представленный на рисунке \ref{img:client-main}, позволяет загрузить расписание и выполнить его конвертацию.

\addimgh{client_main}{0.8}{Главный экран приложения}{img:client-main}

После конвертации расписания, появляется возможность добавить его в список расписаний или скачать конвертированные данные.
При добавлении расписания оно сохраняется в локальное хранилище и будет доступно после перезагрузки приложения.

Выбрав расписание, будет доступен интерфейс просмотра событий расписания, представленный на рисунке \ref{img:client-schedule}.

\addimgh{client_schedule}{0.8}{Интерфейс просмотра расписания}{img:client-schedule}

На данном экране помимо просмотра событий имеется возможность выбора отображаемого временного промежутка, а также выбор периода, в течении которого будут группироваться события.

\subsubsection{Использование приложения}

Реализованное приложение способно представлять различные расписания в общем виде, скрывая дополнительные вычисления.

Для людей использующих комплексные расписания приложение позволяет сфокусироваться на планировании с учетом предстоящих событий не отвлекаясь на дополнительные вычисления.

Для тех, кто использует несколько расписаний приложение предоставляет единую точку доступа к их информации, в едином удобном виде.

\clearpage