\subsection{Реализация управления генераторами}

Текущий процесс представления расписания подразумевает активную работу с итерируемыми сущностями такими как:
\begin{easylist}
  & Временной период
  & Состояния расписания
  & События расписания
\end{easylist}

В спецификации \hyperlink{es6}{ES6} имеется специальный вид функций называемый генераторами \cite{js-generator}.
Данные функции могут приостанавливать свой контекст исполнения, возвращать промежуточный результат и далее возобновлять его в произвольный момент.
Генераторы позволяют гибко контролировать процесс вычисления и инкапсулировать логики обработки.

Например в генераторе временного периода может быть реализована логика переходов от одной даты к другой и фильтрации определенных дат.
Таким образом можно организовать вычисление состояний для минимально необходимого количества моментов времени.

Однако спецификации JS нет удобных аналогов методов массивов для работы с генераторами в функциональном стиле.
Для это была реализована библиотека iterator-wrapper \cite{iterator-wrapper}.

Она предоставляет следующие методы для работы с генераторами:
\begin{easylist}
  & <<wrapIterable>> позволяет обернуть итератор таким образом, что бы при вычислении промежуточного результата итератора, порождался бы новый итератор с промежуточным результатом итератора в качестве начального значения и управление передавалось ему, генератору
  & <<restrictIterable>> позволяет прервать итерирование по достижению указанного условия
  & <<mapIterable>> преобразует каждый элемент итератора с помощью указанной функции отображения
  & <<filterIterable>> позволяет фильтровать значения итератора
  & <<reduceIterable>> позволяет собирать значения итератора в группы на основе функций сепарации и объединения
\end{easylist}

\begin{lstlisting}[caption={Пример работы функции <<reduceIterable>>},label={lst:reduce-iterable}]
function * generator (count: number) {
  for (let i = 0; i < count; i++) {build-act
    yield i
  }
}

function separator (acc: number[]) {
  return acc.length < 4
}

function reducer (acc: number[], val: number) {
  return acc.concat(val)
}

const gen = generator(12)
const reduced = reduceIterable(gen, separator, reducer, [])
console.log(...reduced)
// [0, 1, 2, 3], [4, 5, 6, 7], [8, 9, 10, 11]
\end{lstlisting}