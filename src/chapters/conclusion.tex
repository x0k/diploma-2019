\anonsection{Заключение}

Планирование - оптимальное распределение ресурсов для достижения поставленных целей, деятельность, связанная с постановкой целей и действий в будущем.
Для упрощения данного процесса были изобретены расписания.
Эффективное использование расписания позволяет улучшить процесс планирования.

Эффективность расписания можно оценить через скорость его интерпретации человеком.
Экспресс опрос студентов показал, что около половины респондентов готовы сразу перейти на использование подобной программы после демонстрации ее функционала. Остальные предпочли предварительно получить отзывы о ее работе от других пользователей.
Лишь небольшой процент предпочел использовать “старый метод” работы с фотографией расписания.

В процессе работы были выполнены следующие задачи:
\begin{easylist}
  & Анализ предметной области
  & Моделирование процесса интерпретации расписания
  & Определен способ реализации алгоритма автоматизированной интерпретации
  & Спроектирована структура приложения для представления комплексных расписаний
  & Разработан формат расписания пригодный для интерпретации
  & Реализован алгоритм представления расписания
\end{easylist}

Все поставленные на данном этапе задачи были решены, следовательно цель курсового проекта можно считать достигнутой.