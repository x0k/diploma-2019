\anonsection{Введение}

Расписание - часть процесса планирования с которым сталкивается каждый.
Далеко не всегда для понимания расписания достаточно знания текущей даты, что порой становится актуальной проблемой для людей использующих комплексное расписание.

В качестве объекта исследования выступают расписания, а предметом является представление комплексных расписаний.

Целью данного курсового проекта является проектирование приложения для представления комплексных расписаний с использованием Progressive Web Application подхода.

Для достижения поставленной цели необходимо:
\begin{easylist}[itemize]
  & Выполнить анализ предметной области
  & Произвести моделирование процесса
  & Выбрать оптимальные способы реализации
  & Произвести кодирование
\end{easylist}

Выполнение данных целей позволит получить инструмент предоставляющий возможность представлять комплексные расписания в удобном для восприятия человеком виде.

Используемый инструментарий:
\begin{easylist}[itemize]
  & \LaTeX - набор макрорасширений системы компьютерной вёрстки \TeX
  & Draw.io - векторный редактор
  & Git - система контроля версий
  & Node.js - среда исполнения JavaScript кода
  & Visual Studio Code - редактор исходного кода
\end{easylist}

\clearpage
