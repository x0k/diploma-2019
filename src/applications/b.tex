\appsection{Приложение Б} \hypertarget{app:B}{\label{app:B}}

\centering{Структура транзитного формата данных}

\begin{tblh}{Структура блока <<TDateTimePeriod>>}{dateTimePeriod-block}
  \begin{tabularx}{\textwidth}{| p{3cm} | p{6cm} | X |}
  \hline Наименование & Тип    & Описание                                 \\
  \hline start        & number & Дата начала в формате UNIX Epoch time    \\
  \hline end          & number & Дата окончания в формате UNIX Epoch time \\
  \hline
  \end{tabularx}
\end{tblh}

\begin{tblh}{Структура блока <<IRule>>}{rule-block}
  \begin{tabularx}{\textwidth}{| p{3cm} | p{6cm} | X |}
  \hline Наименование & Тип      & Описание                                                                     \\
  \hline id           & number   & Идентификатор правила                                                        \\
  \hline expression   & any[]    & Интерпретируемое выражение                                                   \\
  \hline require?     & string[] & Список идентификаторов правил, от которых зависит вычисление данного правила \\
  \hline
  \end{tabularx}
\end{tblh}

\begin{tblh}{Структура блока <<IEventOptions>>}{eventOptions-block}
  \begin{tabularx}{\textwidth}{| p{3cm} | p{6cm} | X |}
  \hline Наименование    & Тип               & Описание               \\
  \hline year?           & number | number[] & Ограничение года       \\
  \hline month?          & number | number[] & Включающие ограничения \\
  \hline date?           & number | number[] & Включающие ограничения \\
  \hline day?            & number | number[] & Включающие ограничения \\
  \hline hour?           & number | number[] & Включающие ограничения \\
  \hline minute?         & number | number[] & Включающие ограничения \\
  \hline
  \end{tabularx}
\end{tblh}

\begin{tblh}{Структура блока <<IEventPeriods>>}{eventPeriods-block}
  \begin{tabularx}{\textwidth}{| p{3cm} | p{6cm} | X |}
  \hline Наименование    & Тип             & Описание                                          \\
  \hline dateTimePeriod? & TDateTimePeriod & Полный временной период                           \\
  \hline datePeriod?     & TDateTimePeriod & Временной период с указанием года, месяца и числа \\
  \hline timePeriod?     & TDateTimePeriod & Временной период с указанием часа и минут         \\
  \hline
  \end{tabularx}
\end{tblh}

\clearpage