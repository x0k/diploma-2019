%\documentclass[xetex,mathserif,serif,t]{beamer}
% Если хотим соотношение сторон 16:9
\documentclass[aspectratio=169,xetex,mathserif,serif,t]{beamer}

%%% Преамбула
\usetheme{MasterThesis} % Тема

%%% Работа с русским языком
\usepackage{fontspec,xunicode,xltxtra}

\defaultfontfeatures{Ligatures=TeX} % Для преобразования -- и ---
\setmainfont{PT Sans}
\setmonofont{PT Mono}

%%% Beamer по-русски
\newtheorem{rtheorem}{Теорема}
\newtheorem{rproof}{Доказательство}
\newtheorem{rexample}{Пример}

%%% Дополнительная работа с математикой
\usepackage{amsmath,amsfonts,amssymb,amsthm,mathtools} % AMS
%\usepackage{icomma} % "Умная" запятая: $0,2$ --- число, $0, 2$ --- перечисление

%%% Свои команды
\DeclareMathOperator{\sgn}{\mathop{sgn}}

%%% Перенос знаков в формулах (по Львовскому)
\newcommand*{\hm}[1]{#1\nobreak\discretionary{}
{\hbox{$\mathsurround=0pt #1$}}{}}

%%% Работа с картинками
\usepackage{graphicx}    % Для вставки рисунков
\graphicspath{{../images/}} % Каталоги с картинками
\setlength\fboxsep{3pt}  % Отступ рамки \fbox{} от рисунка
\setlength\fboxrule{1pt} % Толщина линий рамки \fbox{}
\usepackage{wrapfig}     % Обтекание рисунков текстом

%%% Работа с таблицами
\usepackage{array,tabularx,tabulary,booktabs} % Дополнительная работа с таблицами
\usepackage{longtable}  % Длинные таблицы
\usepackage{multirow}   % Слияние строк в таблице

%%% Другие пакеты
\usepackage{lastpage} % Узнать, сколько всего страниц в документе
\usepackage{soulutf8} % Модификаторы начертания
\usepackage{csquotes} % Еще инструменты для ссылок
\usepackage{multicol} % Несколько колонок
\usepackage{etoolbox} % Логические операторы

%%% Картинки
\usepackage{tikz,calc,ifthen,color}

%%% Листинги
\usepackage{listings}

\lstset{
    string=[s]{"}{"},
    stringstyle=\color{blue},
    comment=[l]{:},
    commentstyle=\color{black},
}

\lstdefinelanguage{js}{
  keywords={typeof, new, true, false, catch, function, return, null, catch, switch, var, if, in, while, do, else, case, break, export, extends},
  keywordstyle=\color{blue}\bfseries,
  ndkeywords={class, export, boolean, throw, implements, import, this, interface},
  ndkeywordstyle=\color{darkgray}\bfseries,
  identifierstyle=\color{black},
  sensitive=false,
  comment=[l]{//},
  morecomment=[s]{/*}{*/},
  commentstyle=\color{purple}\ttfamily,
  stringstyle=\color{red}\ttfamily,
  morestring=[b]',
  morestring=[b]"
}

\title{Инструментарий для представления комплексных расписаний на базе веб-платформы}
\author{Красильников Р. Б.}
\date{2019}

\begin{document}
%%% Содержимое слайдов

% Титульный слайд
\frame[plain]{\titlepage}

%-------------------------------------------------------------------------------

\section{Введение}

\begin{frame}
\frametitle{\insertsection}
\framesubtitle{Цели и задачи}

\textbf{Цель:}
\begin{itemize}
    \item Проектирование приложения для представления комплексных расписаний
\end{itemize}

\textbf{Задачи:}
\begin{itemize}
    \item Анализ предметной области
    \item Определение способа реализации процесса
    \item Проектирование архитектуры приложения
    \item Реализация приложения
\end{itemize}

\end{frame}

%-------------------------------------------------------------------------------

\section{Progressive Web Application}

\begin{frame}
\frametitle{\insertsection}

Группа веб-приложений использующих современные возможности веб-платформы.

\vspace{5mm}

\textbf{Преимущества классических веб-приложений:}
\begin{itemize}
    \item Доступность
    \item Универсальность
\end{itemize}

\textbf{Недостатки классических веб-приложений:}
\begin{itemize}
    \item Производительность
    \item Ограниченность среды исполнения кода
\end{itemize}

\end{frame}

%-------------------------------------------------------------------------------

\section{Комплексное расписание}

\begin{frame}
\frametitle{\insertsection}

\begin{figure}
    \center
    \includegraphics[height=60mm]{schedule_types}
\end{figure}
\end{frame}

%-------------------------------------------------------------------------------

\section{Пример комплексного расписания}

\begin{frame}
\frametitle{\insertsection}

\begin{figure}
    \center
    \includegraphics[width=\linewidth]{monday}
\end{figure}
\end{frame}

%-------------------------------------------------------------------------------

\section{Модель процесса планирования}

\begin{frame}
\frametitle{\insertsection}
\framesubtitle{AS-IS}

\begin{figure}
    \center
    \includegraphics[height=70mm]{as_is}
\end{figure}
\end{frame}

%-------------------------------------------------------------------------------

\section{Модель процесса планирования}

\begin{frame}
\frametitle{\insertsection}
\framesubtitle{TO-BE}

\begin{figure}
    \center
    \includegraphics[height=60mm]{to_be}
\end{figure}
\end{frame}

%-------------------------------------------------------------------------------

\section{Результат представления расписания}

\begin{frame}
\frametitle{\insertsection}

\begin{figure}
    \center
    \includegraphics[height=60mm]{tasks}
\end{figure}
\end{frame}

%-------------------------------------------------------------------------------

\section{Что такое расписание?}

\begin{frame}
\frametitle{\insertsection}

\begin{figure}
    \center
    \includegraphics[width=80mm]{date_mapper}
\end{figure}
\end{frame}

%-------------------------------------------------------------------------------

\section{Сущности приложения}

\begin{frame}
\frametitle{\insertsection}

\begin{figure}
    \center
    \includegraphics[width=\linewidth]{class_diagram}
\end{figure}
\end{frame}

%-------------------------------------------------------------------------------

\section{Конвертор}

\begin{frame}[fragile]
\frametitle{\insertsection}

\minipage{0.35\textwidth}
    \begin{lstlisting}[basicstyle=\tiny]
{
    "data": {
        "subject": "Math",
        "type": "Lab",
        "teacher": "Ivan",
        "room": "123"
    },
    "options": {
        "includes": {
            "day": [1, 3]
        }
    },
    "rules": {
        "includes": {
            "call": [6, 7]
        }
    }
}
    \end{lstlisting}
\endminipage\hfill
\minipage{0.6\textwidth}
    \begin{lstlisting}[basicstyle=\tiny]
[
    "@case", [
        "@and", [
            "$>>", "@get", [ "day" ], "@includes", [ [1, 3] ],
            "$>>", "@get", [ "call" ], "@includes", [ [6, 7] ]
        ], "123",
        "@default", [ false ] ],
]
    \end{lstlisting}
\endminipage

\end{frame}

%-------------------------------------------------------------------------------

\section{Генерация временного промежутка}

\begin{frame}[fragile]
\frametitle{\insertsection}

\begin{figure}
    \center
    \includegraphics[height=60mm]{time_structure}
\end{figure}
\end{frame}

%-------------------------------------------------------------------------------

\section{Вычислитель состояний}

\begin{frame}
\frametitle{\insertsection}

\vspace{1cm}

\begin{figure}
    \center
    \includegraphics[width=\linewidth]{event_build_order}
\end{figure}
\end{frame}

%-------------------------------------------------------------------------------

\section{Группировщик}

\begin{frame}
\frametitle{\insertsection}

\vspace{1cm}

\begin{figure}
    \center
    \includegraphics[width=\linewidth]{grouper}
\end{figure}
\end{frame}

%-------------------------------------------------------------------------------

\section{Применение}

\begin{frame}
\frametitle{\insertsection}

\begin{figure}[!htb]
    \minipage{0.61\textwidth}
        \includegraphics[width=\linewidth]{client_schedule}
    \endminipage\hfill
    \minipage{0.34\textwidth}
        \includegraphics[width=\linewidth]{client_mobile}
    \endminipage
\end{figure}

\end{frame}

%-------------------------------------------------------------------------------

\section{Заключение}

\begin{frame}
\frametitle{\insertsection}

\textbf{Были решены следующие задачи:}
\begin{itemize}
    \item Произведен анализ предметной области
    \item Определен способа реализации
    \item Спроектирована архитектура приложения
    \item Реализовано приложение
\end{itemize}

\end{frame}

%-------------------------------------------------------------------------------

\section{Спасибо за внимание}

\begin{frame}
    \frametitle{\insertsection}

    \textbf{Тема:}
    \begin{itemize}
        \item Проектирование приложения для представления комплексных расписаний с использованием подхода Progressive Web Application
    \end{itemize}

    \textbf{Автор:}
    \begin{itemize}
        \item Красильников Роман Борисович
    \end{itemize}

\end{frame}

\end{document}