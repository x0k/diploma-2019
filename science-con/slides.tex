%%% Содержимое слайдов

\frame[plain]{\titlepage} % Титульный слайд

%-------------------------------------------------------------------------------

\section{Основные понятия}

\begin{frame}
\frametitle{\insertsection}
\framesubtitle{Комплексное расписание}

\begin{figure}
    \center
    \includegraphics[height=72mm]{schedule_types}
\end{figure}
\end{frame}

%-------------------------------------------------------------------------------

\section{Основные понятия}

\begin{frame}
\frametitle{\insertsection}
\framesubtitle{Комплексное расписание}

\begin{figure}
    \center
    \includegraphics[width=\linewidth]{monday}
\end{figure}
\end{frame}

%-------------------------------------------------------------------------------

\section{Основные понятия}

\begin{frame}
\frametitle{\insertsection}
\framesubtitle{Веб-платформа}

\textbf{Преимущества:}
\begin{itemize}
    \item Распространенность
    \item Кроссплатформенность
    \item Опыт разработки
\end{itemize}

\textbf{Недостатки:}
\begin{itemize}
    \item Производительность
    \item Ограниченность среды исполнения кода
\end{itemize}

\end{frame}

%-------------------------------------------------------------------------------

\section{Основные понятия}

\begin{frame}
\frametitle{\insertsection}
\framesubtitle{Представление расписания}

\begin{figure}
    \center
    \includegraphics[width=\linewidth]{monday}
\end{figure}
\end{frame}

%-------------------------------------------------------------------------------

\section{Основные понятия}

\begin{frame}
\frametitle{\insertsection}
\framesubtitle{Представление расписания}

\begin{figure}
    \center
    \includegraphics[height=60mm]{tasks}
\end{figure}
\end{frame}

%-------------------------------------------------------------------------------

\section{Наивный подход}

\begin{frame}
\frametitle{\insertsection}
\framesubtitle{Модель данных}

\begin{figure}
    \center
    \includegraphics[height=60mm]{erd}
\end{figure}
\end{frame}

%-------------------------------------------------------------------------------

\section{Наивный подход}

\begin{frame}
\frametitle{\insertsection}
\framesubtitle{Интерфейс ввода данных}

\begin{figure}
    \center
    \includegraphics[height=60mm]{shemaker}
\end{figure}
\end{frame}

%-------------------------------------------------------------------------------

\section{Наивный подход}

\begin{frame}
\frametitle{\insertsection}
\framesubtitle{Интерфейс приложения}

\begin{figure}[!htb]
    \minipage{0.32\textwidth}
        \includegraphics[width=\linewidth]{sheview}
    \endminipage\hfill
    \minipage{0.32\textwidth}
        \includegraphics[width=\linewidth]{sheview_menu}
    \endminipage\hfill
    \minipage{0.32\textwidth}
        \includegraphics[width=\linewidth]{sheview_groups}
    \endminipage
\end{figure}
\end{frame}

%-------------------------------------------------------------------------------

\section{Что такое расписание?}

\begin{frame}
\frametitle{\insertsection}

\begin{figure}
    \center
    \includegraphics[width=70mm]{date_mapper}
\end{figure}
\end{frame}

%-------------------------------------------------------------------------------

\section{Представление расписания}

\begin{frame}[fragile]
\frametitle{\insertsection}
\framesubtitle{Генерация функции}
\minipage{0.35\textwidth}
    \begin{lstlisting}[basicstyle=\tiny]
        {
            "data": {
                "subject": "Предмет",
                "type": "лаб.",
                "teacher": "Преподаватель",
                "room": "251/1"
            },
            "options": {
                "includes": {
                    "day": [1, 3]
                }
            },
            "rules": {
                "includes": {
                    "call": [6, 7]
                }
            }
        }
    \end{lstlisting}
\endminipage\hfill
\minipage{0.6\textwidth}
    \begin{lstlisting}[basicstyle=\tiny]
        [
          "@case", [
            "@and", [
                "$>>", "@get", [ "day" ], "@includes", [ [1, 3] ],
                "$>>", "@get", [ "call" ], "@includes", [ [6, 7] ]
            ],
            "251/1",
            "@default", [ false ] ],
        ]
    \end{lstlisting}
\endminipage

\end{frame}

%-------------------------------------------------------------------------------

\section{Представление расписания}

\begin{frame}[fragile]
\frametitle{\insertsection}
\framesubtitle{Временной интервал}

\begin{lstlisting}[basicstyle=\tiny, language=js]
interface IDictionary<T> {
  [key: string]: T
}

interface IConstraint {
    step?: number | TExpression;
    expression?: TExpression;
}

interface IConstraints extends IDictionary<IConstraint | undefined> {
    [YEAR]?: IConstraint
    [MONTH]?: IConstraint
    [DATE]?: IConstraint
    [HOUR]?: IConstraint
    [MINUTE]?: IConstraint
}
\end{lstlisting}
\end{frame}

%-------------------------------------------------------------------------------

\section{Представление расписания}

\begin{frame}
\frametitle{\insertsection}
\framesubtitle{Генератор событий}

\vspace{2cm}

\begin{figure}
    \center
    \includegraphics[width=\linewidth]{event_build_order}
\end{figure}
\end{frame}

%-------------------------------------------------------------------------------

\section{Представление расписания}

\begin{frame}
\frametitle{\insertsection}
\framesubtitle{Группировщик событий}

\vspace{1cm}

\begin{figure}
    \center
    \includegraphics[width=\linewidth]{grouper}
\end{figure}
\end{frame}

%-------------------------------------------------------------------------------

\section{Применение}

\begin{frame}
\frametitle{\insertsection}

\begin{figure}[!htb]
    \minipage{0.64\textwidth}
        \includegraphics[width=\linewidth]{core_desktop}
    \endminipage\hfill
    \minipage{0.315\textwidth}
        \includegraphics[width=\linewidth]{core_mobile}
    \endminipage
\end{figure}
\end{frame}

%-------------------------------------------------------------------------------

\section{Спасибо за внимание}

\begin{frame}
    \frametitle{\insertsection}
    \textbf{Тема:} Инструментарий для представления комплексных расписаний на базе веб-платформы \\
    \textbf{Автор:} Красильников Роман Борисович
\end{frame}

