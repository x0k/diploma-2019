%%% Пользовательские функции %%%

%% Рисунки %%

% Добавление одного рисунка
\newcommand{\addimg}[4]{
    \begin{figure}
        \centering
        \includegraphics[width=#2\linewidth]{#1}
        \caption{#3} \label{#4}
    \end{figure}
}

% Добавление рисунка в указанное место
\newcommand{\addimgh}[4]{
    \begin{figure}[H]
        \centering
        \includegraphics[width=#2\linewidth]{#1}
        \caption{#3} \label{#4}
    \end{figure}
}

%% Таблицы %%

% \let\oldtabular\tabular 
% \renewcommand{\tabular}{\footnotesize\oldtabular}

% Таблицы
\newenvironment{tbl}[2]
{
    \begin{table}
    \small
    \hyphenpenalty=0
    \caption{#1} \label{#2}
}
{
    \end{table}
}

% Таблица в указанном месте
\newenvironment{tblh}[2]
{
    \begin{table}[H]
    \small
    \hyphenpenalty=0
    \caption{#1} \label{#2}
}
{
    \end{table}
}

% Verbatim окружение с маленьким шрифтом

\newenvironment{code}
{\small\Verbatim}
{\endVerbatim}

%% Секции %%

% Секции без номеров (введение, заключение...), вместо section*{}
\newcommand{\anonsection}[1]{
    \phantomsection % Корректный переход по ссылкам в содержании
    \paragraph{\centerline{\boldtitle{#1}}}
    \addcontentsline{toc}{section}{#1}
}

% Секция для аннотации (она не включается в содержание)
\newcommand{\annotation}[1]{
    \paragraph{\centerline{\boldtitle{#1}}}
}

% Секция для списка иллюстративного материала
\newcommand{\lof}{
    \phantomsection
    \listoffigures
    \addcontentsline{toc}{section}{\listfigurename}
}

% Секция для списка табличного материала
\newcommand{\lot}{
    \phantomsection
    \listoftables
    \addcontentsline{toc}{section}{\listtablename}
}

% Секции для приложений
\newcommand{\appsection}[1]{
    \phantomsection
    \paragraph{\centerline{\textbf{#1}}}
    \addcontentsline{toc}{section}{#1}
}