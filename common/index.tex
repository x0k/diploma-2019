%%% Преамбула %%%

\newcommand{\thetitle}{Проектирование и разработка веб-приложения для представления комплексных расписаний}
\newcommand{\theauthor}{Красильников Роман Борисович}

% Русский язык
\usepackage{polyglossia}
\setdefaultlanguage{russian}

\usepackage{listings} % Оформление исходного кода
\lstset{
    basicstyle=\small\ttfamily, % Размер и тип шрифта
    breaklines=true,            % Перенос строк
    tabsize=2,                  % Размер табуляции
    frame=single,               % Рамка
    literate={--}{{-{}-}}2,     % Корректно отображать двойной дефис
    literate={---}{{-{}-{}-}}3  % Корректно отображать тройной дефис
}

% Шрифты, xelatex
\defaultfontfeatures{Ligatures=TeX}
\setmainfont{Times New Roman}
\newfontfamily\cyrillicfont{Times New Roman}
\setmonofont{FreeMono} % Моноширинный шрифт для оформления кода

% Абзацы и списки
\usepackage{enumerate}   % Тонкая настройка списков
\usepackage[ampersand]{easylist} % Простой синтаксис для списков
\usepackage{indentfirst} % Красная строка после заголовка
\usepackage{float}       % Расширенное управление плавающими объектами
\usepackage{multirow}    % Сложные таблицы

% Пути к каталогам с изображениями
\usepackage{graphicx} % Вставка картинок и дополнений
\graphicspath{{images/}}

% Формат подрисуночных записей
\usepackage{chngcntr}

% Гиперссылки
\usepackage{hyperref}
\hypersetup{
    colorlinks, urlcolor={black}, % Все ссылки черного цвета, кликабельные
    linkcolor={black}, citecolor={black}, filecolor={black},
    pdfauthor={\theauthor},
    pdftitle={\thetitle}
}

% Оформление библиографии и подрисуночных записей через точку
\makeatletter
\renewcommand*{\@biblabel}[1]{\hfill#1.}
\renewcommand*\l@section{\@dottedtocline{1}{1em}{1em}}
\renewcommand{\thefigure}{\arabic{figure}} % Формат рисунка номер
\renewcommand{\thetable}{\arabic{table}}   % Формат таблицы номер
\def\redeflsection{\def\l@section{\@dottedtocline{1}{0em}{10em}}}
\makeatother

\renewcommand{\baselinestretch}{1.4} % Полуторный межстрочный интервал
\parindent 1.27cm % Абзацный отступ

\sloppy             % Избавляемся от переполнений
\hyphenpenalty=10000 % Частота переносов
\clubpenalty=10000  % Запрещаем разрыв страницы после первой строки абзаца
\widowpenalty=10000 % Запрещаем разрыв страницы после последней строки абзаца

% Отступы у страниц
\usepackage{geometry}
\geometry{left=3cm}
\geometry{right=3cm}
\geometry{top=2cm}
\geometry{bottom=2cm}

% Списки
\usepackage{enumitem}
\setlist[enumerate,itemize]{leftmargin=12.7mm} % Отступы в списках

\makeatletter
    \AddEnumerateCounter{\asbuk}{\@asbuk}{м)}
\makeatother
\setlist{nolistsep}                           % Нет отступов между пунктами списка
\renewcommand{\labelitemi}{--}                % Маркер списка --
\renewcommand{\labelenumi}{\asbuk{enumi})}    % Список второго уровня
\renewcommand{\labelenumii}{\arabic{enumii})} % Список третьего уровня

% Содержание
\usepackage{tocloft}
\renewcommand{\cfttoctitlefont}{\hspace{0.38\textwidth}\MakeTextUppercase} % СОДЕРЖАНИЕ
\renewcommand{\cftsecfont}{\hspace{0pt}}            % Имена секций в содержании не жирным шрифтом
\renewcommand\cftsecleader{\cftdotfill{\cftdotsep}} % Точки для секций в содержании
\renewcommand\cftsecpagefont{\mdseries}             % Номера страниц не жирные
\setcounter{tocdepth}{2}                            % Глубина оглавления, до subsubsection

% Подсчет содержимого

\usepackage{totcount}
\regtotcounter{figure}
\regtotcounter{table}
\newtotcounter{citenum}
\def\oldcite{}
\let\oldcite=\bibcite
\def\bibcite{\stepcounter{citenum}\oldcite}

% Список иллюстративного материала
\renewcommand{\cftloftitlefont}{\hspace{0.17\textwidth}\MakeTextUppercase}
\renewcommand{\cftfigfont}{Рисунок }
\addto\captionsrussian{\renewcommand\listfigurename{Список иллюстративного материала}}

% Список табличного материала
\renewcommand{\cftlottitlefont}{\hspace{0.2\textwidth}\MakeTextUppercase}
\renewcommand{\cfttabfont}{Таблица }
\addto\captionsrussian{\renewcommand\listtablename{Список табличного материала}}

% Нумерация страниц посередине сверху
\usepackage{fancyhdr}
\pagestyle{fancy}
\fancyhf{}
\cfoot{\textrm{\thepage}}
\fancyheadoffset{0mm}
\fancyfootoffset{0mm}
\setlength{\headheight}{17pt}
\renewcommand{\headrulewidth}{0pt}
\renewcommand{\footrulewidth}{0pt}
\fancypagestyle{plain}{
    \fancyhf{}
    \cfoot{\textrm{\thepage}}
}

% Формат подрисуночных надписей
\usepackage{caption}
\DeclareCaptionLabelSeparator{defffis}{. } % Разделитель
\captionsetup[figure]{justification=centering, labelsep=defffis, format=plain} % Подпись рисунка по центру
\captionsetup[table]{justification=RaggedLeft, labelsep=defffis, format=plain, singlelinecheck=false} % Подпись таблицы слева
\addto\captionsrussian{\renewcommand{\figurename}{Рисунок}} % Имя фигуры

% Заголовки секций в оглавлении в верхнем регистре
\usepackage{textcase}
\makeatletter
\let\oldcontentsline\contentsline
\def\contentsline#1#2{
    \expandafter\ifx\csname l@#1\endcsname\l@section
        \expandafter\@firstoftwo
    \else
        \expandafter\@secondoftwo
    \fi
    {\oldcontentsline{#1}{\MakeTextUppercase{#2}}}
    {\oldcontentsline{#1}{#2}}
}
\makeatother

% Оформление заголовков
\usepackage[compact,explicit]{titlesec}
\titleformat{\section}{}{}{0mm}{\centering{ГЛАВА~\thesection.~\MakeTextUppercase{#1}}}
\titleformat{\subsection}[block]{\vspace{1em}}{}{12.5mm}{\thesubsection.~#1}
\titleformat{\subsubsection}[block]{}{\hspace{12.5mm}}{0mm}{#1}
\titleformat{\paragraph}[block]{\normalsize}{}{12.5mm}{\MakeTextUppercase{#1}}

% Библиография: отступы и межстрочный интервал
\makeatletter
\renewenvironment{thebibliography}[1]
    {\section*{\refname}
        \list{\@biblabel{\@arabic\c@enumiv}}
           {\settowidth\labelwidth{\@biblabel{#1}}
            \leftmargin\labelsep
            \itemindent 16.7mm
            \@openbib@code
            \usecounter{enumiv}
            \let\p@enumiv\@empty
            \renewcommand\theenumiv{\@arabic\c@enumiv}
        }
        \setlength{\itemsep}{0pt}
    }
\makeatother

\usepackage{lastpage} % Подсчет количества страниц
\setcounter{page}{1}  % Начало нумерации страниц

%%% Пользовательские функции %%%

%% Рисунки %%

% Добавление одного рисунка
\newcommand{\addimg}[4]{
    \begin{figure}
        \centering
        \includegraphics[width=#2\linewidth]{#1}
        \caption{#3} \label{#4}
    \end{figure}
}

% Добавление рисунка в указанное место
\newcommand{\addimgh}[4]{
    \begin{figure}[H]
        \centering
        \includegraphics[width=#2\linewidth]{#1}
        \caption{#3} \label{#4}
    \end{figure}
}

%% Таблицы %%

% \let\oldtabular\tabular 
% \renewcommand{\tabular}{\footnotesize\oldtabular}

% Таблицы
\newenvironment{tbl}[2]
{
    \begin{table}
    \small
    \hyphenpenalty=0
    \caption{#1} \label{#2}
}
{
    \end{table}
}

% Таблица в указанном месте
\newenvironment{tblh}[2]
{
    \begin{table}[H]
    \small
    \hyphenpenalty=0
    \caption{#1} \label{#2}
}
{
    \end{table}
}

% Verbatim окружение с маленьким шрифтом

\newenvironment{code}
{\small\Verbatim}
{\endVerbatim}

%% Секции %%

% Секции без номеров (введение, заключение...), вместо section*{}
\newcommand{\anonsection}[1]{
    \phantomsection % Корректный переход по ссылкам в содержании
    \paragraph{\centerline{\boldtitle{#1}}}
    \addcontentsline{toc}{section}{#1}
}

% Секция для аннотации (она не включается в содержание)
\newcommand{\annotation}[1]{
    \paragraph{\centerline{\boldtitle{#1}}}
}

% Секция для списка иллюстративного материала
\newcommand{\lof}{
    \phantomsection
    \listoffigures
    \addcontentsline{toc}{section}{\listfigurename}
}

% Секция для списка табличного материала
\newcommand{\lot}{
    \phantomsection
    \listoftables
    \addcontentsline{toc}{section}{\listtablename}
}

% Секции для приложений
\newcommand{\appsection}[1]{
    \phantomsection
    \paragraph{\centerline{\textbf{#1}}}
    \addcontentsline{toc}{section}{#1}
}